\documentclass[12pt]{article}
\usepackage[utf8]{inputenc}

\title{Sample document}
\author{Antoine Martin}
\date{August 2022}

\begin{document}

\begin{titlepage}
\maketitle
\end{titlepage}

Logoden biniou degemer mat an penn ar bed devezh, grizilh kozh goleiñ an paeañ
koad don ilin e, Plouha war koustañ bras gwech vazh gwern. Hep yaouank riskl
pevarzek drezon sklaer bleun nijal gwin, prad start kenavo veur benveg kluchañ
amprevan sec’hañ liv, divalav laerezh doujañ boan lammat waz gar. Ganet ennon
bremañ degemer fourchetez trubard aotre skeudenn keniterv, geot familh drezomp
gwin houad niz kroaz beajourien divskouarn, hep talvoudus blot Ar Vouster
voutailh hini gwrierez. Digalon bugel Oskaleg nec’h hi bod muioc’h etre dilun,
pehini askell Groe Naoned enep hed lagad Roazhon kouevr, Ar Releg-Kerhuon ebeul
beajiñ nerzh dihuniñ warnomp c’houlz. Vrozh lizherenn wec’h da kenavo boutailh
kiger e enno, leskiñ zo vrec’h eme servijañ ebeul kegin koll harzhal, perc’henn
Montroulez ur poent e fentigelloù gouzout. Touellañ poazhañ c’hastell ha pobl
pesketaer Plouzane voger anezhi, sizailh forn o gounez korf moan nevez nizez
keit, Abbarez naet goulenn kriz morzhol kontadenn mil. Laerezh kelc’hiek
gwelloc’h hon reas yec’hed daou glac’har kanañ, c’hof tud bann dehou mintin aval
vloaz kenwerzh aes, levr ribl karantez kegin arvar gourc’hemennoù degouezhout.
Roazhon montr gasoni muzell toto butuniñ yezh abred kuzuliañ karrezek ijinañ war
strad tal, ur Skrigneg perc’henn lunedoù brezhoneg pounner nerzh fall oferenn
tach kustum. Gwaskañ evezh c’hroc’hen gwez kousket traonienn stourm pont hep
tavarn diwezhat pegañ marteze an, yaouank gontell eil gwerenn Pornizhan botez
bolz zoken Kemper vantell Briad.

\end{document}
